%%%%%%%%%%%%%%%%%%%%%%%%%%%%%%%%%%%%%%%%
% datoteka diploma-FRI-vzorec.tex
%
% vzorčna datoteka za pisanje diplomskega dela v formatu LaTeX
% na UL Fakulteti za računalništvo in informatiko
%
% na osnovi starejših verzij vkup spravil Franc Solina, maj 2021
% prvo verzijo je leta 2010 pripravil Gašper Fijavž
%
% za upravljanje z literaturo ta vezija uporablja BibLaTeX
%
% svetujemo uporabo Overleaf.com - na tej spletni implementaciji LaTeXa ta vzorec zagotovo pravilno deluje
%

\documentclass[a4paper,12pt,openright]{book}
%\documentclass[a4paper, 12pt, openright, draft]{book}  Nalogo preverite tudi z opcijo draft, ki pokaže, katere vrstice so predolge! Pozor, v draft opciji, se slike ne pokažejo!
 
\usepackage[utf8]{inputenc}   % omogoča uporabo slovenskih črk kodiranih v formatu UTF-8
\usepackage[slovene,english]{babel}    % naloži, med drugim, slovenske delilne vzorce
\usepackage[pdftex]{graphicx}  % omogoča vlaganje slik različnih formatov
\usepackage{fancyhdr}          % poskrbi, na primer, za glave strani
\usepackage{amssymb}           % dodatni matematični simboli
\usepackage{amsmath}           % eqref, npr.
\usepackage{hyperxmp}
\usepackage[hyphens]{url}
\usepackage{csquotes}
\usepackage[pdftex, colorlinks=true,
						citecolor=black, filecolor=black, 
						linkcolor=black, urlcolor=black,
						pdfproducer={LaTeX}, pdfcreator={LaTeX}]{hyperref}

\usepackage{color}
\usepackage{soul}

\usepackage[
backend=biber,
style=numeric,
sorting=nty,
]{biblatex}


\addbibresource{literatura.bib} %Imports bibliography file


%%%%%%%%%%%%%%%%%%%%%%%%%%%%%%%%%%%%%%%%
%	DIPLOMA INFO
%%%%%%%%%%%%%%%%%%%%%%%%%%%%%%%%%%%%%%%%
\newcommand{\ttitle}{Spletna aplikacija za glasbene festivale}
\newcommand{\ttitleEn}{Web application for music festivals}
\newcommand{\tsubject}{\ttitle}
\newcommand{\tsubjectEn}{\ttitleEn}
\newcommand{\tauthor}{Vasja Škarabot}
\newcommand{\tkeywords}{glasba, festival, socialno omrežje, spletna aplikacija}
\newcommand{\tkeywordsEn}{music, festival, social network, web application}

%%%%%%%%%%%%%%%%%%%%%%%%%%%%%%%%%%%%%%%%
%	HYPERREF SETUP
%%%%%%%%%%%%%%%%%%%%%%%%%%%%%%%%%%%%%%%%
\hypersetup{pdftitle={\ttitle}}
\hypersetup{pdfsubject=\ttitleEn}
\hypersetup{pdfauthor={\tauthor}}
\hypersetup{pdfkeywords=\tkeywordsEn}

%%%%%%%%%%%%%%%%%%%%%%%%%%%%%%%%%%%%%%%%
% postavitev strani
%%%%%%%%%%%%%%%%%%%%%%%%%%%%%%%%%%%%%%%%  

\addtolength{\marginparwidth}{-20pt} % robovi za tisk
\addtolength{\oddsidemargin}{40pt}
\addtolength{\evensidemargin}{-40pt}

\renewcommand{\baselinestretch}{1.3} % ustrezen razmik med vrsticami
\setlength{\headheight}{15pt}        % potreben prostor na vrhu
\renewcommand{\chaptermark}[1]%
{\markboth{\MakeUppercase{\thechapter.\ #1}}{}} \renewcommand{\sectionmark}[1]%
{\markright{\MakeUppercase{\thesection.\ #1}}} \renewcommand{\headrulewidth}{0.5pt} \renewcommand{\footrulewidth}{0pt}
\fancyhf{}
\fancyhead[LE,RO]{\sl \thepage} 
%\fancyhead[LO]{\sl \rightmark} \fancyhead[RE]{\sl \leftmark}
\fancyhead[RE]{\sc \tauthor}              % dodal Solina
\fancyhead[LO]{\sc Diplomska naloga}     % dodal Solina


\newcommand{\BibLaTeX}{{\sc Bib}\LaTeX}
\newcommand{\BibTeX}{{\sc Bib}\TeX}

%%%%%%%%%%%%%%%%%%%%%%%%%%%%%%%%%%%%%%%%
% naslovi
%%%%%%%%%%%%%%%%%%%%%%%%%%%%%%%%%%%%%%%%  

\newcommand{\autfont}{\Large}
\newcommand{\titfont}{\LARGE\bf}
\newcommand{\clearemptydoublepage}{\newpage{\pagestyle{empty}\cleardoublepage}}
\setcounter{tocdepth}{1}	      % globina kazala

%%%%%%%%%%%%%%%%%%%%%%%%%%%%%%%%%%%%%%%%
% konstrukti
%%%%%%%%%%%%%%%%%%%%%%%%%%%%%%%%%%%%%%%%  
\newtheorem{izrek}{Izrek}[chapter]
\newtheorem{trditev}{Trditev}[izrek]
\newenvironment{dokaz}{\emph{Dokaz.}\ }{\hspace{\fill}{$\Box$}}


%%%%%%%%%%%%%%%%%%%%%%%%%%%%%%%%%%%%%%%%%%%%%%%%%%%%%%%%%%%%%%%%%%%%%%%%%%%%%%%
%% PDF-A
%%%%%%%%%%%%%%%%%%%%%%%%%%%%%%%%%%%%%%%%%%%%%%%%%%%%%%%%%%%%%%%%%%%%%%%%%%%%%%%

%%%%%%%%%%%%%%%%%%%%%%%%%%%%%%%%%%%%%%%% 
% define medatata
%%%%%%%%%%%%%%%%%%%%%%%%%%%%%%%%%%%%%%%% 
\def\Title{\ttitle}
\def\Author{\tauthor, vs7015@student.uni-lj.si}
\def\Subject{\ttitleEn}
\def\Keywords{\tkeywordsEn}

%%%%%%%%%%%%%%%%%%%%%%%%%%%%%%%%%%%%%%%% 
% \convertDate converts D:20080419103507+02'00' to 2008-04-19T10:35:07+02:00
%%%%%%%%%%%%%%%%%%%%%%%%%%%%%%%%%%%%%%%% 
\def\convertDate{%
    \getYear
}

{\catcode`\D=12
 \gdef\getYear D:#1#2#3#4{\edef\xYear{#1#2#3#4}\getMonth}
}
\def\getMonth#1#2{\edef\xMonth{#1#2}\getDay}
\def\getDay#1#2{\edef\xDay{#1#2}\getHour}
\def\getHour#1#2{\edef\xHour{#1#2}\getMin}
\def\getMin#1#2{\edef\xMin{#1#2}\getSec}
\def\getSec#1#2{\edef\xSec{#1#2}\getTZh}
\def\getTZh +#1#2{\edef\xTZh{#1#2}\getTZm}
\def\getTZm '#1#2'{%
    \edef\xTZm{#1#2}%
    \edef\convDate{\xYear-\xMonth-\xDay T\xHour:\xMin:\xSec+\xTZh:\xTZm}%
}

%\expandafter\convertDate\pdfcreationdate 

%%%%%%%%%%%%%%%%%%%%%%%%%%%%%%%%%%%%%%%%
% get pdftex version string
%%%%%%%%%%%%%%%%%%%%%%%%%%%%%%%%%%%%%%%% 
\newcount\countA
\countA=\pdftexversion
\advance \countA by -100
\def\pdftexVersionStr{pdfTeX-1.\the\countA.\pdftexrevision}


%%%%%%%%%%%%%%%%%%%%%%%%%%%%%%%%%%%%%%%%
% XMP data
%%%%%%%%%%%%%%%%%%%%%%%%%%%%%%%%%%%%%%%%  
\usepackage{xmpincl}
%\includexmp{pdfa-1b}

%%%%%%%%%%%%%%%%%%%%%%%%%%%%%%%%%%%%%%%%
% pdfInfo
%%%%%%%%%%%%%%%%%%%%%%%%%%%%%%%%%%%%%%%%  
\pdfinfo{%
    /Title    (\ttitle)
    /Author   (\tauthor, vs7015@student.uni-lj.si)
    /Subject  (\ttitleEn)
    /Keywords (\tkeywordsEn)
    /ModDate  (\pdfcreationdate)
    /Trapped  /False
}

%%%%%%%%%%%%%%%%%%%%%%%%%%%%%%%%%%%%%%%%
% znaki za copyright stran
%%%%%%%%%%%%%%%%%%%%%%%%%%%%%%%%%%%%%%%%  

\newcommand{\CcImageCc}[1]{%
	\includegraphics[scale=#1]{cc_cc_30.pdf}%
}
\newcommand{\CcImageBy}[1]{%
	\includegraphics[scale=#1]{cc_by_30.pdf}%
}
\newcommand{\CcImageSa}[1]{%
	\includegraphics[scale=#1]{cc_sa_30.pdf}%
}

%%%%%%%%%%%%%%%%%%%%%%%%%%%%%%%%%%%%%%%%%%%%%%%%%%%%%%%%%%%%%%%%%%%%%%%%%%%%%%%
%%%%%%%%%%%%%%%%%%%%%%%%%%%%%%%%%%%%%%%%%%%%%%%%%%%%%%%%%%%%%%%%%%%%%%%%%%%%%%%

\begin{document}
\selectlanguage{slovene}
\frontmatter
\setcounter{page}{1} %
\renewcommand{\thepage}{}       % preprečimo težave s številkami strani v kazalu

%%%%%%%%%%%%%%%%%%%%%%%%%%%%%%%%%%%%%%%%
%naslovnica
 \thispagestyle{empty}%
   \begin{center}
    {\large\sc Univerza v Ljubljani\\%
%      Fakulteta za elektrotehniko\\% za študijski program Multimedija
%      Fakulteta za upravo\\% za študijski program Upravna informatika
      Fakulteta za računalništvo in informatiko\\%
%      Fakulteta za matematiko in fiziko\\% za študijski program Računalništvo in matematika
     }
    \vskip 10em%
    {\autfont \tauthor\par}%
    {\titfont \ttitle \par}%
    {\vskip 3em \textsc{DIPLOMSKO DELO\\[5mm]         % dodal Solina za ostale študijske programe
    VISOKOŠOLSKI STROKOVNI ŠTUDIJSKI PROGRAM\\ PRVE STOPNJE\\ RAČUNALNIŠTVO IN INFORMATIKA}\par}%
%     UNIVERZITETNI  ŠTUDIJSKI PROGRAM\\ PRVE STOPNJE\\ RAČUNALNIŠTVO IN INFORMATIKA}\par}%
%    INTERDISCIPLINARNI UNIVERZITETNI\\ ŠTUDIJSKI PROGRAM PRVE STOPNJE\\ MULTIMEDIJA}\par}%
%    INTERDISCIPLINARNI UNIVERZITETNI\\ ŠTUDIJSKI PROGRAM PRVE STOPNJE\\ UPRAVNA INFORMATIKA}\par}%
%    INTERDISCIPLINARNI UNIVERZITETNI\\ ŠTUDIJSKI PROGRAM PRVE STOPNJE\\ RAČUNALNIŠTVO IN MATEMATIKA}\par}%
    \vfill\null%
% izberite pravi habilitacijski naziv mentorja!
    {\large \textsc{Mentor}: viš. pred. dr. Aljaž Zrnec\par}%
    {\vskip 2em \large Ljubljana, \the\year \par}%
\end{center}
% prazna stran
%\clearemptydoublepage      
% izjava o licencah itd. se izpiše na hrbtni strani naslovnice

%%%%%%%%%%%%%%%%%%%%%%%%%%%%%%%%%%%%%%%%
%copyright stran
%%%%%%%%%%%%%%%%%%%%%%%%%%%%%%%%%%%%%%%%
\newpage
\thispagestyle{empty}

\vspace*{5cm}
{\small \noindent
To delo je ponujeno pod licenco \textit{Creative Commons Priznanje avtorstva-Deljenje pod enakimi pogoji 2.5 Slovenija} (ali novej\v so razli\v cico).
To pomeni, da se tako besedilo, slike, grafi in druge sestavine dela kot tudi rezultati diplomskega dela lahko prosto distribuirajo,
reproducirajo, uporabljajo, priobčujejo javnosti in predelujejo, pod pogojem, da se jasno in vidno navede avtorja in naslov tega
dela in da se v primeru spremembe, preoblikovanja ali uporabe tega dela v svojem delu, lahko distribuira predelava le pod
licenco, ki je enaka tej.
Podrobnosti licence so dostopne na spletni strani \href{http://creativecommons.si}{creativecommons.si} ali na Inštitutu za
intelektualno lastnino, Streliška 1, 1000 Ljubljana.

\vspace*{1cm}
\begin{center}% 0.66 / 0.89 = 0.741573033707865
\CcImageCc{0.741573033707865}\hspace*{1ex}\CcImageBy{1}\hspace*{1ex}\CcImageSa{1}%
\end{center}
}

\vspace*{1cm}
{\small \noindent
Izvorna koda diplomskega dela, njeni rezultati in v ta namen razvita programska oprema je ponujena pod licenco GNU General Public License,
različica 3 (ali novejša). To pomeni, da se lahko prosto distribuira in/ali predeluje pod njenimi pogoji.
Podrobnosti licence so dostopne na spletni strani \url{http://www.gnu.org/licenses/}.
}

\vfill
\begin{center} 
\ \\ \vfill
{\em
Besedilo je oblikovano z urejevalnikom besedil \LaTeX.}
\end{center}

% prazna stran
\clearemptydoublepage

%%%%%%%%%%%%%%%%%%%%%%%%%%%%%%%%%%%%%%%%
% stran 3 med uvodnimi listi
\thispagestyle{empty}
\
\vfill

\bigskip
\noindent\textbf{Kandidat:} Vasja Škarabot\\
\noindent\textbf{Naslov:} Spletna aplikacija za glasbene festivale\\
% vstavite ustrezen naziv študijskega programa!
\noindent\textbf{Vrsta naloge:} Diplomska naloga na visokošolskem programu prve stopnje Računalništvo in informatika \\
% izberite pravi habilitacijski naziv mentorja!
\noindent\textbf{Mentor:} viš. pred. dr. Aljaž Zrnec\\

\bigskip
\noindent\textbf{Opis:}\\
Besedilo teme diplomskega dela študent prepiše iz študijskega informacijskega sistema, kamor ga je vnesel mentor. 
V nekaj stavkih bo opisal, kaj pričakuje od kandidatovega diplomskega dela. 
Kaj so cilji, kakšne metode naj uporabi, morda bo zapisal tudi ključno literaturo.

\bigskip
\noindent\textbf{Title:} Web application for music festivals

\bigskip
\noindent\textbf{Description:}\\
opis diplome v angleščini

\vfill



\vspace{2cm}

% prazna stran
\clearemptydoublepage

% zahvala
\thispagestyle{empty}\mbox{}\vfill\null\it%
\noindent
Na tem mestu zapišite, komu se zahvaljujete za pomoč pri izdelavi diplomske naloge oziroma pri vašem študiju nasploh. Pazite, da ne boste koga pozabili. Utegnil vam bo zameriti. Temu se da izogniti tako, da celotno zahvalo izpustite.
\rm\normalfont

% prazna stran
\clearemptydoublepage


%%%%%%%%%%%%%%%%%%%%%%%%%%%%%%%%%%%%%%%%
% kazalo
\pagestyle{empty}
\def\thepage{}% preprečimo težave s številkami strani v kazalu
\tableofcontents{}


% prazna stran
\clearemptydoublepage

%%%%%%%%%%%%%%%%%%%%%%%%%%%%%%%%%%%%%%%%
% seznam kratic

\chapter*{Seznam uporabljenih kratic}

\noindent\begin{tabular}{p{0.11\textwidth}|p{.39\textwidth}|p{.39\textwidth}}    % po potrebi razširi prvo kolono tabele na račun drugih dveh!
  {\bf kratica} & {\bf angleško}                              & {\bf slovensko} \\ \hline
  {\bf API}      & Application Programming Interface               & Aplikacijski programski vmesnik \\
  {\bf DBMS} & database management system & sistem za upravljanje podatkovnih baz \\
  {\bf SVM}   & support vector machine              & metoda podpornih vektorjev \\
%  \dots & \dots & \dots \\
\end{tabular}


% prazna stran
\clearemptydoublepage

%%%%%%%%%%%%%%%%%%%%%%%%%%%%%%%%%%%%%%%%
% povzetek
\phantomsection
\addcontentsline{toc}{chapter}{Povzetek}
\chapter*{Povzetek}

\noindent\textbf{Naslov:} \ttitle
\bigskip

\noindent\textbf{Avtor:} \tauthor
\bigskip

%\noindent\textbf{Povzetek:} 
\noindent 
Diplomska naloga predstavlja razvoj in implementacijo spletne aplikacije za glasbene festivale. Glavni namen aplikacije je, uporabnikom olajšati
dostop do vseh pomembnih informacij, ter jim omogočiti medsebojno komunikacijo in preko le te graditi skupnosti s pomočjo forumov in klepetov. 

Spletna aplikacija uporabniku omogoča pregled nad festivali, vključno z iskanjem prenočišč v bližini ter navodili za pot do prizorišča, forum na katerem lahko uporabniki
objavljajo novice, delijo nasvete ter postavljajo morebitna vprašanja, ki se jim porajajo, ter klepeti, katere lahko uporabniki uporabijo z namenom razpravljanja in dogovarjanja za skupni obisk. 

Za implementacijo čelnega dela smo uporabili ogrodje Nuxt.js, za zaledni del pa Django z razširitvijo Rest Framework in podatkovno bazo PostgreSQL. Vse skupaj poganjamo v Dockerju.
Zemljevidi so implementirani s pomočjo platforme Mapbox, realnočasovno komunikacijo v klepetih omogoča Pusher, slike pa shranjujemo s pomočjo Firebase shrambe.

\bigskip

\noindent\textbf{Ključne besede:} \tkeywords.
% prazna stran
\clearemptydoublepage

%%%%%%%%%%%%%%%%%%%%%%%%%%%%%%%%%%%%%%%%
% abstract
\phantomsection
\selectlanguage{english}
\addcontentsline{toc}{chapter}{Abstract}
\chapter*{Abstract}

\noindent\textbf{Title:} \ttitleEn
\bigskip

\noindent\textbf{Author:} \tauthor
\bigskip

%\noindent\textbf{Abstract:} 
\noindent 
The thesis presents the development and implementation of a web application for music festivals. The main purpose of the application is to make it easier for users
access to all relevant information, and to allow users to communicate with each other and build communities through forums and chats. 

The web application provides the user with an overview of the festivals, including a search for nearby accommodation and directions to the venue, a forum where users can
post news, share tips and ask any questions they may have, and chat rooms which users can use to discuss and arrange to visit together. 

For the front-end implementation, we used the Nuxt.js framework, and for the back-end we used Django with the Rest Framework extension and the PostgreSQL database. 
We run everything inside Docker containers. Maps are implemented using the Mapbox platform, real-time communication in chats is enabled by Pusher, and images are stored using the Firebase Storage.

\bigskip

\noindent\textbf{Keywords:} \tkeywordsEn.
\selectlanguage{slovene}
% prazna stran
\clearemptydoublepage

%%%%%%%%%%%%%%%%%%%%%%%%%%%%%%%%%%%%%%%%
\mainmatter
\setcounter{page}{1}
\pagestyle{fancy}

\chapter{Uvod}
\section{Motivacija}

Vsako leto veliko ljudi obišče kakšen glasbeni festival. Pridobivanje informacij in samo načrtovanje obiska pa lahko kar hitro postane nadležno, 
in posledično lahko hitro pokvari uporabniško izkušnjo, že pred samim obiskom festivala.
Najprej se uporabnik odloči za enega izmed festivalov. Ponavadi to stori na podlagi različnih dejavnikov, kot so cena, lokacija in datum poteka, veliko pa 
pomenijo tudi predhodne izkušnje ostalih. Problem pa je, da je potrebno vse te informacije iskati po različnih spletnih straneh saj rešitve,
s katero bi lahko pridobili vse informacije na enem mestu ni. Poleg tega bi bilo smiselno implementirati rešitev, kjer si lahko uporabniki medeseboj pomagajo, 
s tem da delijo predhodne izkušnje, novosti in mnenja v obliki foruma ali pa morda celo dobijo nove prijatelje, v živo razpravljajo in mogoče nekoč celo skupaj
obiščejo kakšen festival s pomočjo klepetov v živo. Z razvojem namenske spletne aplikacije bi obiskovalcem festivalov lahko pomagali tako, da bi se za vsako pomoč lahko obrnili na našo rešitev, 
istočasno pa bi se z medsebojno pomočjo gradile skupnosti za glasbene festivale.

% TODO poglej si kaj je ref{}


\chapter{Obstoječe rešitve}

V uporabi trenutno ni nobene podobne rešitve, zaradi česar se obiskovalci festivalov poslužujejo različnih aplikacij in spletnih strani. 
Za postavljanje vprašanj in objavljanje novic sta to Reddit ali pa Tripadvisor. Ti rešitvi nista primarno namenjeni glasbenim festivalom, 
zato težko najdemo koristne podatke, saj vsebujeta le forume za večje festivale.
Poleg tega mora vsak uporabljati več aplikacij za različne namene in zato nimamo vsega na eni platformi. Uporabnik mora tako npr. za iskanje informacij o festivalu najprej na spletno stran festivala,
nato prebira različne forume (Reddit, Tripadvisor), zraven tega še išče prenočišča na Bookingu, potem pa se še na tretji platformi z ostalimi pogovarja/dogovarja (npr. Messenger).
V tem poglavju bodo opisane naštete alternative, njihove prednosti in slabosti.

\section{Tripadvisor}
Tripadvisor je spletna platforma za iskanje prenočišč \cite{tripadvisor}. Je ena večjih platform namenjenih planiranju potovanj, ki uporabnikom omogoča tudi podajanje kritik in ocen.

\begin{itemize}
    \item Prednosti:
        \begin{itemize}
            \item Velik nabor uporabnikov in posledično kritik in ocen za prenočišča in razne dogodke.
            \item Širok nabor funkcionalnosti. Poleg osnovnih funkcionalnosti še orodja za načrtovanje poti, vodniki in priporočila.
        \end{itemize}
    \item Slabosti:
        \begin{itemize}
            \item Vsebuje le najpopularnejše festivale.
            \item Ocene bolj osredotočene na nastanitve in potovanje kot na sam festival.
            \item Ne vsebuje nekaterih pomembnih informacij, kot so datumi, informacije o cenah in izvajalcih.
            \item Uporabnik lahko z ostalimi uporabniki komunicira le posredno preko foruma. Za neposredno komunikacijo se mora poslužiti drugih platform.
        \end{itemize}
\end{itemize}

\section{Reddit}
Reddit je forumsko socialno omrežje, kjer lahko uporabniki objavljajo, ostali pa nato te objave komentirajo in glasujejo \cite{reddit}. Objave so razvrščene v podstrani imenovane subredditi oz. skupnosti.
Na Redditu najdemo skupnosti za nekatere glasbene festivale, kjer se uporabniki obveščajo o novicah o festivalu in podajajo različna vprašanja. Je pa teh skupnosti precej malo, še tiste ki so, 
pa so po večini skupnosti največjih glasbenih festivalov na svetu (Coachella, Tomorrowland, UMF).

\begin{itemize}
    \item Prednosti:
        \begin{itemize}
            \item Pri večjih festivalih ogromen nabor vprašanj, razprav, nasvetov in novic.
            \item Možnost realno-časovne komunikacije z ostalimi.
        \end{itemize}
    \item Slabosti:
        \begin{itemize}
            \item Večino osnovnih informacij mora uporabnik pridobiti na ostalih straneh.
            \item Težko je najti skupnosti, ki niso namenjene le največjim festivalom.
            \item Čeprav lahko uporabniki komunicirajo v realnem času, lahko to počnejo le v privatnih skupinah, ki niso del subredditov.
        \end{itemize}
\end{itemize}

\section{Woov}
Woov je namenska aplikacija za obiskovalce glasbenih festivalov \cite{woov}. Povezani so z več kot 1000 dogodki po 40 državah, uporabnikom pa omogočajo vpogled v časovnice, zemljevid prizorišč in komunikacijo z ostalimi uporabniki.
Namen aplikacije je pomagati obiskovalcem, ki so že na prizorišču.

\begin{itemize}
    \item Prednosti:
        \begin{itemize}
            \item Uporabniki lahko aplikacijo prenesejo na telefon.
            \item Pomembne informacije kot so datumi, prizorišče, nastopajoči izvajalci.
            \item Uporabniki si lahko pomagajo z zemljevidom, ki vključuje lokacije stojnic, wc-jev, odrov in polnilnih postaj.
            \item Realno-časovna komunikacija.
        \end{itemize}
    \item Slabosti:
        \begin{itemize}
            \item Dogodki so običajno dodani nekaj dni pred začetkom festivala. Posledično si uporabnik za planiranje ne more dosti pomagati.
            \item Aplikacija včasih deluje nelogična za nove uporabnike, saj so nekatere funkcionalnosti precej skrite.
            \item Uporabniki ne morejo podati svojih mnenj.
            \item Aplikaciji lahko prispevajo le partnerji. To ni nujno slabo, vendar omeji aplikacije le na določene festivale.
        \end{itemize}
\end{itemize}

\chapter{Zajem zahtev}
Eden ključnih delov pri postopku razvoja programske rešitve je zajem zahtev, saj omogoča lažje razumevanje problema, ter zagotavlja, 
da sistem izpolnjuje vsa pričakovanja in usmerja celoten razvojni proces od načrtovanja do testiranja.
V osnovi jih delimo na funkcionalne in nefunkcionalne zahteve.

\section{Funkcionalne zahteve}
Sistem mora uporabniku omogočiti:
\begin{itemize}
    \item Registracijo in prijavo.
    \item Pregled nad glasbenimi festivali z možnostjo iskanja in filtriranja.
    \item Dodajanje festivalov in dodelitev moderatorskih pravic.
    \item Izpis podrobnosti glasbenega festivala, vključno z datumi, žanri in povezavo do uradne spletne strani.
    \item Pogled na zemljevid z lokacijo glasbenega festivala, ki se lahko razširi v navodila za pot.
    \item Iskanje prenočišč v okolici glasbenega festivala glede na podane parametre.
    \item Branje objav z možnostjo filtriranja in iskanja na forumu za določen festival.
    \item Objavljanje na forumih.
    \item Všečkanje in komentiranje objav.
    \item Dodajanje javnih klepetov za festivale.
    \item Pošiljanje sporočil v klepetih.
    \item Naročanje na obvestila za klepete ob novih sporočilih.
\end{itemize}

Moderatorji pa lahko povrh vsega še:
\begin{itemize}
    \item Urejajo podatke za svoje festivale.
    \item Brišejo neprimerne objave.
    \item Brišejo neprimerne komentarje.
\end{itemize}

\section{Nefunkcionalne zahteve}
\begin{itemize}
    \item Javni klepeti ter obvestila naj omogočajo realnočasovno komunikacijo. Uporabniku se morajo nova sporočila takoj prikazati na zaslonu.
\end{itemize}

\section{Diagram primerov uporabe}


\chapter{Načrtovanje sistema}

\chapter{Uporabljene tehnologije in orodja}
% TODO - directions api za mapbox, geocoding nevem al je api al ne?
Odločili smo se za uporabo modernih, dobro poznanih in pogosto uporabljenih tehnologij in orodij. Za čelni del smo uporabili
Nuxt \cite{nuxtintro}, ki temelji na ogrodju Vue.js \cite{vuejs}. Za zaledni del smo izbrali Django \cite{django} s knjižnico Django REST Framework \cite{drf}, 
ki močno olajša razvoj API vmesnikov in upravljanje s podatkovno bazo. Za podatkovno bazo smo uporabili PostgreSQL \cite{postgre}.
Realnočasovno komunikacijo smo omogočili s pomočjo Pusher Channels \cite{pusher}. Zemljevidi 
in navodila za pot so osnovani na platformi Mapbox \cite{mapbox}. Podatkovno bazo ter Django strežnik smo poganjali v Dockerju \cite{docker}.
Vse te tehnologije in orodja ter njihov namen in uporabo v naši rešitvi smo podrobneje predstavili v naslednjem poglavju.

Poleg omenjenih glavnih tehnologij smo sicer uporabili še Vuetify.js za oblikovanje spletne strani \cite{vuetify},
ter Git \cite{git} in GitHub \cite{github} za nadzor različic, za shranjevanje slik smo se odločili za Firebase Storage \cite{firebase}, za urejanje kode pa 
smo uporabljali Visual Studio Code \cite{vscode}. Poleg tega smo uporabili nekaj Python knjižnic, med njimi BeautifulSoup4 \cite{bs4} za strganje podatkov iz bookinga,
ter Djoser \cite{djoser}, ki ponuja vnaprej definirane končne točke za registracijo, prijavo, odjavo in aktivacijo uporabniškega računa. 

\section{Nuxt.js}
Nuxt.js (na kratko Nuxt) je brezplačno odprtokodno orodje za razvoj čelnega dela spletnih aplikacij z Vue.js. V naši aplikaciji smo ga uporabili zato, ker 
omogoča hiter, enostaven, predvsem pa učinkovit razvoj aplikacij. 

Omogoča pisanje varne kode s samodejno generiranimi tipi, brez potrebe po učenju TypeScript \cite{typescript}.
Poleg tega Nuxt nudi tudi SSR (Server-Side Rendering), ki omogoča hitrejše čase nalaganja strani in SEO (Search Engine Optimization). Pri tem strežnik po začetni zahtevi odjemalca 
pošlje v celoti renderirano stran nazaj odjemalcu. JavaScript na strani odjemalca nato omogoči, da statična stran postane interaktivna Vue.js aplikacija (hidracija) \cite{hydration}.

Ena večjih prednosti je tudi enostavna konfiguracija poti na spletni strani, saj Nuxt samodejno razbere strukturo glede na vsebino mape pages/. 
V Vue.js npr. je za vsako novo dodano stran to potrebno storiti ročno.

\section{Django in DRF}
Django je Python ogrodje za razvoj spletnih aplikacij, ustvarjeno z namenom da uporabniku omogoči hiter razvoj, s tem, da vsebuje širok nabor orodij,
ki poskrbijo za pogosto uporabljena opravila spletnega razvoja (avtentikacija, administracija, upravljanje z podatkovno bazo...).
Temelji na vzorcu Model-View-Template, ki je v osnovi zelo podoben bolj znanemu Model-View-Controller. Model skrbi za podatke in vzdržuje povezavo s podatkovno bazo.
View sprejema zahtevke in vrača odgovore, Template pa definira strukturo in postavitev strani, tipično z uporabo DTL (Django Template Language) \cite{mvt}.

Django REST Framework je razširitev za Django, posebej namenjena za gradnjo RESTful API-jev. V naši aplikaciji smo DRF v uporabili predvsem zaradi 
enostavnega dela s podatkovno bazo, enostavne serializacije podatkov (pretvorba kompleksnih podatkovnih tipov kot so Django modeli in poizvedbe v
format JSON ali XMl) in že vnaprej pripravljenih pogledov za nekatere osnovne CRUD operacije, 
ki jih z lahkoto lahko priredimo za naše namene. Poleg tega omogoča tudi enostavno pisanje pravic za dostop do API končnih točk.

\section{PostgreSQL}
Za podatkovno bazo smo uporabili PostgreSQL. PostgreSQL je zmogljiv odprtokoden objektno-relacijski DBMS (sistem za upravljanje s podatkovnimi bazami). Kljub temu, da je
odprtokoden se lahko kosa z ostalimi ponudniki, kot sta npr. Oracle in MySQL. Prav zaradi stroškov, se zelo pogosto uporablja v različnih startupih in za v 
raziskovalne namene \cite{juba2015learning}.

\section{Pusher Channels}
Realnočasovno komunikacijo v klepetih smo zagotovili s pomočjo storitve Pusher Channels. Pusher Channels deluje na tehnologiji WebSockets, ki omogoča vztrajne TCP povezave
med strežnikom in odjemalcem. S tem lahko strežnik in odjemalec dosežeta takojšnjo izmenjavo sporočil na zelo učinkovit način z majhno zakasnitvijo, brez da bi odjemalec
poslal zahtevo za prejem podatkov \cite{websockets}.

Pusher Channels delujejo na modelu objavi/naroči. Aplikacije se lahko naročajo na kanale v sistemu. Ko pride do sprememb, pa sistem objavi spremembo v ta isti kanal.
Vse aplikacije naročene na ta kanal, so potem obveščene \cite{pusher}.

\section{Mapbox}
Mapbox je spletna platforma, ki razvijalcem ponuja širok nabor orodij za delo z zemljevidi. Podatke črpa iz različnih virov, med drugim tudi iz MapStreetBoxa in NASE. 
V naši aplikaciji smo uporabili večino glavnih storitev, ki jih omogoča in sicer \cite{mapboxservices}:
\begin{itemize}
    \item \textbf{Mapbox Maps}, ki omogoča prikaz zemljevidov po meri z lokacijami. V naši aplikaciji prikazuje vse zemljevide.
    \item \textbf{Mapbox Navigation}, ki omogoča iskanje in izrisovanje poti. V naši aplikaciji služi izrisovanju poti od vnešene lokacije do festivala.
    \item \textbf{Geocoding API}, ki spreminja koordinate v naslove in obratno. V naši aplikaciji je uporabljen za človeku prijazen prikaz lokacije.
    \item \textbf{Mapbox Search}, ki omogoča iskanje lokacij. V naši aplikaciji je uporabljen za iskanje začetnih lokacij pri navigaciji.
    \item \textbf{Mapbox GL Geocoder}, ki omogoča interaktivno iskanje po zemljevidu. V naši aplikaciji je na voljo moderatorjem, da lahko na interaktiven način poiščejo lokacijo festivala.
\end{itemize}

\section{Docker}
Docker je platforma, ki omogoča ustvarjanje, izvajanje in nameščanje aplikacij v kontejnerjih/vsebnikih. Temu procesu pravimo kontejnerizacija.
Pri kontejnerizaciji zapakiramo programsko kodo s knjižnicami operacijskega sistema in odvisnostmi, ki so potrebne za izvajanje kode.
S tem ustvarimo lahko izvedljivo datoteko, ki se lahko enako izvaja na kateri koli infrastrukturi \cite{containerization}.
Prednosti uporabe Dockerja je veliko, v našem primeru pa smo ga uporabili predvsem zaradi izolacije okolja in lahke prenosljivosti.
V Dockerju smo poganjali Django aplikacijo ter podatkovno bazo.

\chapter{Razvoj rešitve}

\chapter{Predstavitev aplikacije}

\chapter{Možne izboljšave}

\chapter{Zaključek}

\LaTeX\ bi lahko najbolj preprosto opisali kot programski jezik namenjen oblikovanju besedil.
Tako kot vsak visokonivojski programski jezik ima tudi \LaTeX\  številne ukaze za oblikovanje  besedila in okolja, ki omogočajo strukturiranje besedila.

Vsi \LaTeX ovi ukazi se začnejo z levo poševnico  \verb=\=, okolja pa definiramo bodisi s parom zavitih oklepajev \{ in \} ali z ukazoma \verb=\begin{ }= in   \verb=\end{ }=.
Ukazi imajo lahko tudi argumente, obvezni argumenti so podani v zavitih oklepajih, opcijski argumenti pa v oglatih oklepajih.

Z ukazi torej definiramo naslov in imena avtorjev besedila, poglavja in podpoglavja in po potrebi bolj podrobno strukturiramo besedila na spiske, navedke itd.
Posebna okolja so namenjena zapisu matematičnih izrazov, kratki primeri so v naslednjem poglavju.

Vse besedilne konstrukte lahko poimenujemo in se s pomočjo teh imen nato  kjerkoli v besedilu na njih  tudi sklicujemo.

\LaTeX\ sam razporeja besede v odstavke tako, da optimizira razmike med besedami v celotnem odstavku.
Nov odstavek začnemo tako, da izpustimo v izvornem besedilu prazno vrstico. Da besedilo skoči v novo vrstico pa ukažemo z dvema levima poševnicama.
Število presledkov med besedami v izvornem besedilo ni pomembno.


\chapter{Matematično okolje in sklicevanje na besedilne konstrukte}
\label{ch1}

Matematična ali popolna indukcija je eno prvih orodij, ki jih spoznamo za dokazovanje trditev pri matematičnih predmetih.
\begin{izrek}
\label{iz:1}
Za vsako naravno število $n$ velja
\begin{equation}
n < 2^n.
\label{eq:1}
\end{equation}
\end{izrek}
\begin{dokaz}
Dokazovanje z indukcijo zahteva, da neenakost~\eqref{eq:1} najprej preverimo za najmanjše naravno število -- $0$. 
Res, ker je $0 < 1 = 2^0$, je neenačba~\eqref{eq:1} za $n=0$ izpolnjena.

Sledi indukcijski korak. S predpostavko, da je neenakost~\eqref{eq:1} veljavna pri nekem naravnem številu $n$, je potrebno pokazati, da je ista neenakost v veljavi tudi pri njegovem nasledniku -- naravnem številu $n+1$. 
Računajmo.
\begin{align}
n+1 & < 2^n + 1       \label{eq:2}\\
       & \le 2^n + 2^n \label{eq:3}\\
       & = 2^{n+1}       \nonumber
\end{align}
Neenakost~\eqref{eq:2} je posledica indukcijske predpostavke, neenakost~\eqref{eq:3} pa enostavno dejstvo, da je za vsako naravno število $n$ izraz $2^n$ vsaj tako velik kot 1. 
S tem je dokaz Izreka~\ref{iz:1} zaključen.
\end{dokaz}

Opazimo, da je \LaTeX\ številko izreka podredil številki poglavja.
Na podoben način se lahko s pomočjo ukazov \verb|\label| in \verb|\ref| sklicujemo tudi na druge besedilne konstrukte, kot so med drugim poglavja, podpoglavja in plovke, ki jih bomo spoznali v naslednjem poglavju.


\chapter{Plovke: slike in tabele}
\label{ch2}

Slike in daljše tabele praviloma vključujemo v dokument kot plovke.
Pozicija plovke v končnem izdelku ni pogojena s tekom besedila, temveč z izgledom strani. 
\LaTeX\ bo skušal plovko postaviti samostojno, praviloma na mestu, kjer se pojavi v izvornem besedilu, sicer pa na 
vrhu strani, na kateri se na takšno plovko prvič sklicujemo. 
Pri tem pa bo na vsako stran končnega izdelka želel postaviti tudi sorazmerno velik del besedila. 
V skrajnem primeru, če imamo res preveč plovk na enem mestu besedila, ali če je plovka previsoka, se bo \LaTeX\ odločil za stran popolnoma zapolnjeno s plovkami.

Poleg tega, da na položaj plovke vplivamo s tem, kam jo umestimo v izvorno besedilo, lahko na položaj plovke na posamezni strani prevedenega besedila dodatno vplivamo z opcijami \texttt{here, top} in \texttt{bottom}.
Zelo velike slike je najbolje postaviti na posebno stran z opcijo \texttt{page}.
Skaliranje slik po njihovi širini lahko prilagodimo širini strani tako, da kot enoto za širino uporabimo kar širino strani, npr. \verb=0.5\textwidth= bo raztegnilo sliko na polovico širine strani.
Sliko lahko po potrebi tudi zavrtimo za 90 stopinj in jo razstegnemo na višino strani. 
Tako bodo podrobnosti na sliki lažje berljive in prostor na strani bo bolje izkoriščen.

Na vse plovke se moramo v besedilu sklicevati, saj kot beseda plovka pove, plovke plovejo po besedilu in se ne pojavijo točno tam, kjer nastopajo v izvornem besedilu.
Vendar naj bosta  sklic na plovko v besedilu in sama plovka v oblikovanem besedilu čim bližje skupaj, tako da bralcu ne bo potrebno listati po diplomi. 
Upoštevajte pa, da se naloge tiska dvostransko in da se hkrati vidi dve strani v dokumentu!
Na to, kje se bo slika ali druga plovka pojavila v postavljenem besedilu lahko torej najbolj vplivamo tako, da v izvorni kodi plovko premikamo po besedilu nazaj ali naprej!

Tabele je najbolje oblikovati kar neposredno v \LaTeX u, saj za oblikovanje tabel obstaja zelo fleksibilno okolje \texttt{tabular}.
Slike pa je po drugi strani  pogosto najla\v zje oblikovati oziroma izdelati z drugimi orodji in programi, rezultate shraniti v formatu {\tt .pdf} ali {\tt .jpeg} in nato v \LaTeX u le vključiti ustrezno slikovno datoteko. Za pisanje besed, ki so vključene v slike, uporabite pisave/fonte, ki so čimbolj podobne pisavam v samem besedilu.

Knjižnica \url{https://en.wikibooks.org/wiki/LaTeX/PGF/TikZ} 
pa omo\-go\-ča risanje raznovrstnih grafov neposredno v okolju \LaTeX .

Na vse tabele in slike se moramo v besedilu sklicevati, saj kot plovke v oblikovanem besedilu niso nujno na istem mestu kot v izvornem besedilu.
Pri sklicevanju na slike uporabimo veliko začetnico, npr. ''glej Sliko \ref{pic1}'', saj gre za ime slike.



\section{Formati slik}

V  dokument \LaTeX\ lahko vključimo slike različnih formatov, tako
bitne slike kot vektorske slike. Najbolj primerne so slike v formatu {\tt .pdf}, saj je tudi samo
oblikovano besedilo v tem formatu, in slike v formatu {\tt .jpeg}.
Slika~\ref{pic1} je npr. v formatu {\tt .jpeg}.

\begin{figure}[htb]
\begin{center}
\includegraphics[width=0.7\textwidth]{galaksija}
\end{center}
\caption{Virtualno obogatena skulptura \cite{vodnjak}. Rezultate računalniško generirane animacije z video projektorjem projeciramo na kamnito skulpturo, da ustvarimo vtis, kot da bi po skulpturi polzele vodne kapljice \cite{video,solina2020skulpture}.}
\label{pic1}
\end{figure}



\section{Podnapisi k slikam in tabelam}

Vsaki sliki ali tabeli moramo dodati podnapis, ki na kratko pojasnjuje, kaj je na sliki ali tabeli. 
Če nekdo le prelista diplomsko delo, naj bi že iz slik in njihovih podnapisov lahko na grobo razbral, kakšno temo naloga obravnava.

Če slike povzamemo iz drugih virov, potem se moramo v podnapisu k taki sliki sklicevati na ta vir!


\chapter{Struktura strokovnih besedil}
\label{stroka}

Strokovna besedila imajo ustaljeno strukturo, da bi lahko hitreje in lažje brali in predvsem razumeli taka besedila, saj načeloma vemo vnaprej, 
kje v besedilu se naj bi nahajale določene informacije.

Najbolj osnovna struktura strokovnega besedila je:
\begin{description}
\item[naslov besedila,] ki naj bo sicer kratek, a kljub temu dovolj poveden o vsebini besedila,
\item[imena avtorjev] so običajno navedena po teži prispevka, prvi avtor je tisti, ki je besedilo dejansko pisal, zadnji pa tisti, ki je raziskavo vodil,
\item[kontaktni podatki] -- poleg imena in naslova institucije je potreben vsaj naslov elektronske pošte,
\item[povzetek] je kratko besedilo, ki povsem samostojno povzame vsebino in izpostavi predvsem  glavne rezultate ali zaključke,
\item[ključne besede] so tudi namenjene iskanju vsebin med množico člankov,
\item[uvodno poglavje] uvede bralca v tematiko besedila, razloži kaj je namen besedila, predstavi področje o katerem besedilo piše 
(če temu ni namenjeno v celoti posebno poglavje) ter na kratko predstavi strukturo celotnega besedila,
\item[poglavja] tvorijo zaokrožene celote, ki se po potrebi še nadalje členijo na podpoglavja, namenjena so recimo opisu orodij, 
ki smo jih uporabili pri delu, teoretičnim rezultatom ali predstavitvi rezultatov, ki smo jih dosegli,
\item[zaključek] še enkrat izpostavi glavne rezultate ali ugotovitve, jih primerja z dosedanjimi in morebiti poda tudi ideje za nadaljno delo,
\item[literatura] je seznam vseh virov, na katere smo se pri svojem delu opirali, oziroma smo se na njih sklicevali v svojem besedilu.
\end{description}

Naslove poglavij in podpoglavij izbiramo tako, da lahko bralec že pri prelistavanju diplome in branju naslovov
v grobem ugotovi, kaj je vsebina diplomskega dela.

Strokovna besedila običajno pišemo v prvi osebi množine, v nevtralnem in umirjenem tonu. 
Uporaba sopomenk ni zaželena, saj želimo zaradi lažjega razumevanja za iste pojme vseskozi uporabljati iste besede.
Najpomembnejše ugotovitve je smiselno večkrat zapisati, na primer v povzetku, uvodu, glavnem delu in zaključku.
Vse trditve naj bi temeljile bodisi na lastnih ugotovitvah (izpeljavah, preizkusih, testiranjih) ali pa z navajanjem ustreznih virov.

Največ se lahko naučimo s skrbnim branjem dobrih zgledov takih besedil.


\chapter{Pogoste napake pri pisanju v slovenščini}  % poglavje dodal Solina
\label{slo}

V slovenščini moramo paziti  pri uporabi pridevnikov, ki se ne sklanjajo, kot so npr. kratice. 
Pravilno pišemo ``model CAD'' in \textbf{ne} ``CAD model''!

Pri sklanjanju tujih imen ne uporabljamo vezajev, pravilno je Applov operacijski sistem in \textbf{ne} Apple-ov.

Pika, klicaj in vprašaj so levostični: pred njimi ni presledka, za njimi pa je presledek. 
Klicajev in vprašajev se v strokovnih besedilih načeloma izogibamo. Oklepaji so desnostični in zaklepaji levostični: (takole).

Z narekovaji označujemo premi govor, naslove, citate ali pa z njim dajemo besedam poseben pomen. Narekovaji so stična ločila. Ločimo različne narekovaje, vendar je v
\LaTeX u najbolj enostavno uporabiti ``dvojni narekovaj zgoraj''. Za druge vrste narekovajev je potrebno uvoziti dodatne pakete ali fonte.
Besede lahko vizualno označimo tudi z uporabo drugih pisav  iz iste družine, npr. 
kurzivno in krepko pisavo, vendar pri uporabi teh fontov ne smemo pretiravati.

Vezaj  je levo in desno stičen, npr. \verb=slovensko-angleški slovar= in ga pišemo z enim znakom za pomišljaj.
V slovenščini je presledek pred in po pomišljaju: Pozor -- hud pes! (\verb=Pozor -- hud pes!=).
V angleščini pa je za razliko pomišljaj levo in desno stičen in se v \LaTeX u piše s tremi  pomišljaji: \verb=---=.
S stičnim pomišljajem pa lahko nadomeščamo predlog od \dots do, denimo pri navajanju strani, npr. preberite strani 7--11 (\verb=7--11=).

``Pred ki, ko, ker, da, če vejica skače''. 
To osnovnošolsko pravilo smo v življenju po potrebi uporabljali, dopolnili, morda celo pozabili. 
Pravilo sicer drži, ampak samo če je izpolnjenih kar nekaj pogojev (npr. da so ti vezniki samostojni, enobesedni, ne gre za vrivek itd.).
Povedki so med seboj ločeni z vejicami, razen če so zvezani z in, pa, ter, ne–ne, niti–niti, ali, bodisi, oziroma.
Sicer pa je bolje pisati kratke stavke kot pretirano dolge.

V računalništvu se stalno pojavljajo novi pojmi in nove besede, za katere pogosto še ne obstajajo uveljavljeni slovenski izrazi.
Kadar smo v dvomih, kateri slovenski izraz je primeren, si lahko pomagamo z iskanjem na kakšnem od slovenskih spletnih slovarjev~\cite{slovarji}, še posebej v
\textit{Islovarju} Slovenskega društva Informatika \cite{Islovar} in Slovarju Slovenskega društva za razpoznavanje vzorcev~\cite{sdrv}.
Sicer pa glavni vir za reševanje slovenskih jezikovnih zadreg spletišče \textit{Fran}~\cite{fran}.


\chapter{Koristni nasveti pri pisanju v \LaTeX{u}}
\label{latex}

Programski paket \LaTeX\ je bil prvotno predstavljen v priročniku~\cite{lamport} in je v resnici nadgradnja sistema \TeX\ avtorja Donalda Knutha~\cite{knuth}, 
znanega po svojih knjigah o umetnosti programiranja
ter Knuth-Bendixovem algoritmu~\cite{knuth1983simple}.
\TeX\ in njegove izpeljanke so odprtokodni programi.

Različnih implementacij \LaTeX{}a je cela vrsta.
Za OS X priporočamo TeXShop, za Windows PC pa MikTeX.
Spletna verzija, ki poenostavi sodelovanje pri pisanju, je Overleaf.

Včasih smo si pri pisanju v \LaTeX{}u  pomagali predvsem s tiskanimi pri\-ro\-čni\-ki~\cite{lamport}, danes pa je enostavneje in hitreje, da ob vsakem problemu za pomoč enostavno povprašamo Google, 
saj je na spletu cela vrsta forumov za pomoč pri \TeX iranju.

\LaTeX\ včasih ne zna pravilno deliti slovenskih besed, ki vsebujejo črke s streši\-ca\-mi. 
Če taka beseda štrli preko desnega roba,  lahko \LaTeX{}u pokažemo, kje se tako besedo deli takole: \verb=ra\-ču\-nal\-ni\-štvo=.
Katere vrstice so predolge lahko vidimo tako, da dokument prevedemo s vključeno opcijo \texttt{draft}: \verb=\documentclass[a4paper, 12pt, draft]{book}=.

Predlagamo, da v izvornem besedilu začenjate vsak stavek v novi vrstici, saj \LaTeX\ sam razporeja besede po vrsticah postavljenega besedila. 
Bo pa zato iskanje po izvornem besedilu in popravljanje veliko hitrejše. 
Večina sistemov za \TeX{}iranje sicer omogoča s klikanjem enostavno prestopanje  iz prevedenega besedila na ustrezno mesto v izvornem besedilu in obratno.

Boljšo preglednost dosežemo tako kot pri pisanju programske kode -- z vizualnim urejanjem kode in izpuščanjem praznih vrstic.
Pri spreminjanju in dodajanju izvornega besedila je najbolje pogosto prevajati, da se sproti prepričamo, če so naši nameni pravilno izpolnjeni.

Kadar besedilo, ki je že bilo napisano z nekim vizualnim urejevalnikom (npr. z Wordom), želimo prenesti v \LaTeX, je tudi najbolje to delati postopoma s posameznimi bloki besedila, 
tako da lahko morebitne napake hitro identificiramo in odpravimo.
Za prevajanje Wordovih datotek v \LaTeX\ -- in obratno -- sicer obstajajo prevajalniki, ki pa običajno ne generirajo tako čisto logično strukturo besedila, kot jo sicer \LaTeX\ omogoča.
Hiter in enostaven način prevedbe besedila, ki  zahteva sicer ročne dopolnitve, lahko poteka tudi tako, da besedilo urejeno z vizualnim urejevalnikom najprej shranimo v formatu pdf, 
nato pa to besedilo uvozimo v urejevalnik, kjer urejamo izvorno besedilo v formatu \LaTeX.




\section{Pisave v \LaTeX u}

V  \LaTeX ovem okolju lahko načeloma uporabljamo poljubne pisave. 
Izbira poljubne pisave pa ni tako enostavna kot v vizualnih urejevalnikih besedil.
Posamezne oblikovno medseboj usklajene pisave so običajno združene v družine pisav.
V \LaTeX u se privzeta družina pisav imenuje Computer Modern,
kjer so poleg navadnih črk (roman v \LaTeX u) na voljo tudi kurzivne črke (\textit{italic} v \LaTeX u), 
krepke (\textbf{bold} v \LaTeX u), kapitelke (\textsc{small caps} v \LaTeX u), linearne črke ({\textsf{san serif} v \LaTeX u}), pisava pisalnega stroja (\texttt{typewriter} v \LaTeX u) in nekatere njihove kombinacije, npr. krepke linearne črke 
({\textbf{\textsf{san serif}} v \LaTeX u}).
V istem dokumentu zaradi skladnega izleda uporabljamo običajno le pisave ene družine. 
Pomembna je tudi konsistentna raba večih pisav in da ne pretiravamo z mešanjem različnih pisav.

Ko začenjamo uporabljati \LaTeX, je zato najbolj smiselno uporabljati kar privzete pisave, s katerimi je napisan tudi ta dokument.
Z ustreznimi ukazi  lahko nato preklapljamo med navadnimi, kurzivnimi, krepkimi in drugimi pisavami. 
Zelo enostavna je tudi izbira velikosti črk.
\LaTeX\  odlično podpira večjezičnost, tudi v sklopu istega dokumenta, saj obstajajo pisave za praktično vse jezike, tudi take, ki ne uporabljajo latinskih črk.

Za prikaz programske kode se pogosto uporablja pisava, kjer imajo vse črke enako širino, kot so  črke na mehanskem pisalnem stroju ({\texttt{typewriter} v \LaTeX u}).

Najbolj priročno okolje za pisanje kratkih izsekov programske kode je okolje \texttt{verbatim}, saj ta ohranja vizualno organizacijo izvornega besedila in ima privzeto pisavo pisalnega stroja.

\begin{verbatim}
for (i = 0; i < 100; i++)
   for (j = i; j < 10; j++)
      some_function(i, j);
\end{verbatim}




\chapter{Kaj pa literatura?}
\label{lit}

Kot smo omenili že v uvodu, je pravi način za citiranje literature uporaba \BibLaTeX{a}~\cite{biblatex}. 
\BibLaTeX\ zagotovi, da pri določeni vrsti literature ne izpustimo 
nobene obvezne informacije 
in da vse informacije dosledno navajamo na enak način in po istem vrstnem redu.
\BibLaTeX\ je nadgradnja starejšega sistema \BibTeX. Novejši sistem je bolje prilagojen slovenščini in navajanju spletnih virov. Sicer pa so starejše datoteke \texttt{.bib} kompatibilne z \BibLaTeX om.

Osnovna ideja \BibLaTeX{a} je, da vse informacije o literaturi zapisujemo v posebno datoteko, v našem primeru je to \texttt{literatura.bib}.
Vsakemu viru v tej datoteki določimo simbolično ime.
V  našem primeru je v tej datoteki nekaj najbolj značilnih zvrsti literature, kot so knjige~\cite{lamport}, 
članki v revijah~\cite{leonardo} in zbornikih konferenc~\cite{ciuha2010visualization},
poglavja v knjigah~\cite{poglavje_springer}, 
spletni viri~\cite{slovarji,video}, 
tehnično poročilo~\cite{andersen2012kinect}, 
diplome~\cite{diploma} itd.
Diploma~\cite{diploma} iz leta 1990 je bila prva diploma na tedanji Fakulteti za elektrotehniko in računalništvo, ki je bila oblikovana z \LaTeX om!
Reference, ki so na spletnih straneh arhivirane v elektronski obliki, imajo običajno  \v stevilko DOI (\url{http://dx.doi.org}), ki jo zato tudi vključimo v izpis literature in tako bralcu elektronske verzije naše publikacije ponudimo neposredno povezavo do elektronske kopije te reference.

Po vsaki spremembi pri sklicu na literaturo moramo najprej prevesti izvorno besedilo s prevajalnikom \LaTeX, nato s prevajalnikom  \BibLaTeX, ki ustvari datoteko  {\tt vzorec\_dip\_Seminar.bbl}, in nato še dvakrat s prevajalnikom  \LaTeX.
V okolju Overleaf je to večkratno prevajanje z različnimi prevajalniki uporabniku skrito. Zato tudi začetnim uporabnikom \LaTeX a svetujemo uporabo Overleafa.

Kako se spisek literature nato izpiše (ali so posamezni viri razvrščeni po vrstnem redu sklicevanja, ali po abecedi priimkov prvih avtorjev, ali se imena avtorjev pišejo pred priimki itd.) je odvisno od parametrov paketa \BibLaTeX.
V diplomi bomo uporabili parametre 
\texttt{style=numeric}, kar pomeni, da bodo sklici na literaturo v besedilu označeni z zaporednimi številkami, za vrstni red izpisa referenc pa 
\texttt{sorting=nty}, kar pomeni, da bodo reference urejene po priimkih prvih avtorjev, nato po naslovu reference in nazadnje po letu izdaje~\cite{ctan}.
Zato je potrebno pri določenih zvrsteh literature, ki nima avtorjev, dodati parameter \texttt{key}, ki določi vrstni red vira po abecedi.

Ko začenjamo uporabljati \BibLaTeX\ je lažje, če za urejanje datoteke \texttt{.bib} uporabljamo kar isti urejevalnik kot za urejanje datotek \texttt{.tex}, 
čeprav obstajajo tudi posebni urejevalniki oziroma programi za delo z datotekami \texttt{.bib}.

Le če se  na določen vir v besedilu tudi sklicujemo, se bo ta vir pojavil tudi v spisku literature.
Tako je avtomatično zagotovljeno, da se na vsak vir v seznamu literature tudi sklicujemo v besedilu diplome.
V datoteki \texttt{.bib} imamo sicer lahko veliko več virov za literaturo, kot jih bomo uporabili v diplomi.


\section{Zbiranje virov za seznam literature}

Vire v formatu \texttt{.bib}  lahko enostavno poiščemo in prekopiramo iz spletnih strani založnikov ali različnih akademskih spletnih portalov za iskanje znanstvene literature.
Izvoz  referenc v Google učenjaku še dodatno poenostavimo, če v nastavitvah izberemo \BibTeX\ kot želeni format za izvoz navedb.
Navedbe, ki jih  prekopiramo iz Google učenjaka in drugih podobnih akademskih portalov, moramo pred uporabo nujno preveriti, saj so taki navedki pogosto generirani povsem avtomatično in lahko vsebujejo napačne ali nepopolne podatke.
Najpogosteje je napačen tip publikacije!

Pri sklicevanju na literaturo na koncu stavka moramo paziti, da je pika po ukazu \verb=\cite{ }=.
Da \LaTeX\ ne bi delil vrstico ravno tako, da bi sklic na literaturo v oglatih oklepajih začel novo vrstico, lahko pred sklicem na literaturo dodamo nedeljiv presledek: \verb=~\cite{ }=.

Običajno se v besedilu sklicujemo  na nek vir ali več virov na koncu trdilnega stavka. 
Kadar pa omenimo avtorja nekega vira, pa sklic običajno vstavimo za njegovim priimkom.


Dandanes  se skoraj  vsi pri iskanju informacij vedno najprej lotimo iskanja preko svetovnega spleta.
Rezultati takega iskanja pa so pogosto spletne strani, ki danes obstajajo, jutri pa jih morda ne bo več, ali pa vsaj ne v taki obliki, kot smo jo prebrali.
Smisel navajanja literature pa je, da tudi po dolgih letih nekdo, ki bo bral vašo diplomo, lahko poišče vire, ki jih navajate v diplomi.

Znanstveni rezultati, ki so objavljeni v obliki recenziranih člankov, bodisi v konferenčnih zbornikih, še bolje pa v znanstvenih revijah, so veliko bolj izčiščen in zanesljiv vir informacij, saj
so taki članki šli skozi recenzijske postopke. Predvsem pa so taki članki stabilen vir informacij, saj se načeloma po njihovi objavi ne spreminjajo več.
Skoraj vsi ti članki so dandanes dosegljivi tudi v elektronski obliki, bodisi v arhivih založnikov, univerzitetnih repozitorijih ali tudi na osebnih spletnih straneh njihovih avtorjev.
Zato na svetovnem spletu začenjamo iskati vire za strokovna besedila predvsem preko akademskih spletnih portalov, kot so npr. Google učenjak, Research Gate ali Academia, saj so na teh portalih rezultati iskanja le akademske publikacije.
Če je za dostop do nekega članka potrebno plačati, se obrnemo za pomoč in dodatne informacije na  našo knjižnico.

Za označevanje člankov, ki so na voljo v elektronski obliki, se je v zadnjem času uveljavila oznaka DOI
(\url{https://www.doi.org}), kar močno olajša iskanje teh referenc na spletu.
Založniki tudi starejšim člankom, ki so na voljo v elektronski obliki, za nazaj določajo oznake DOI.
Zato poskusite poiskati ustrezno oznako DOI za vsak članek, ki ga citirate in jo vključite v seznam literature.

Če res ne gre drugače, pa je pomembno, da pri sklicevanju na običajni spletni vir vedno navedemo tudi datum, kdaj smo dostopali do tega vira.

Z uporabo \BibLaTeX{a} je možno natisniti seznam literature posebej za določene vrste referenc, na primer za članke v znanstvenih revijah, članke v konferenčnih zbornikih in poglavja v knjigah, kot je prikazano tudi v tem vzorcu diplome. Pri zbiranju literature si zato prizadevajte čimbolj napolniti sezname teh treh vrst referenc.


\chapter{Skladnost s standardom PDF/A}
\label{PDF}


Elektronsko verzijo diplome je potrebno oddati preko sistema STUDIS v formatu PDF/A ~\cite{howtopdfa,pdfa},
natančneje v formatu PDF/A-1b. 
PDF/A format je namenjen dolgoročnemu arhiviranju elektronskih dokumentov.
Dokument v formatu PDF/A mora vsebovati vse potrebne informacije za prikazovanje in tiskanje dokumenta. To pomeni, da mora dokument  vsebovati vso besedilo, vse slike, fonte in barvne informacije.
Prva verzija standarda PDF/A, to je PDF/A-1 je bil objavljena leta 2005.
Standard PDF/A-1 določa dva nivoja skladnosti: PDF/A-1a in PDF/A-1b.
Nivo a (accesible) mora ustrezati vsem zahtevam standarda.
Nivo b (basic) pa zahteva le, da se ohrani vizualni izgled dokumenta.
Diplome, ki jih je potrebno oddati na sistemu STUDIS, morajo ustrezati nivoju standarda 
PDF/A-1b.

\LaTeX\ in omenjeni format imata še nekaj težav s sobivanjem. 
Paket \texttt{pdfx.sty}, ki naj bi \LaTeX{u} omogočal podporo formatu PDF/A ne deluje vedno
v skladu s pričakovanji. 

Zato raje priporočamo uporabo enega od mnogih spletnih mest, ki omo\-go\-ča\-jo konverzijo pdf datotek v obliko,
ki je skladna s standardom PDF/A-1b, npr. \url{https://pdf.online/pdf-to-pdfa}, kjer je
možno tudi testirati, ali je neka pdf datoteka skladna s tem standardom.

V predlogi so poleg izvornega  dokumenta \texttt{diploma-FRI-vzorec.tex}, še vložena slika \texttt{galaksija.jpeg}, datoteka \texttt{literatura.bib} za uporabljeno literaturo  ter
ikone za licenco Creative Commons.


\chapter{Sklepne ugotovitve}

Uporaba \LaTeX{a} in \BibLaTeX{a} je v okviru Diplomskega seminarja \textbf{obvezna}!
Izbira -- \LaTeX\ ali ne \LaTeX\ -- pri pisanju dejanske diplomske naloge pa je pre\-pu\-šče\-na dogovoru med diplomantom in njegovim mentorjem.

Res je, da so prvi koraki v \LaTeX{}u težavni. 
Ta dokument naj služi kot začetna opora pri hoji.
Pri kakršnihkoli nadaljnih vprašanjih ali napakah pa svetujemo uporabo Googla, saj je spletnih strani za pomoč pri odpravljanju težav pri uporabi \LaTeX{}a ogromno.

Preden diplomo oddate na sistemu STUDIS, še enkrat preverite, če so slovenske besede, ki vsebujejo črke s strešicami,  pravilno deljene in da ne segajo preko desnega roba.
Poravnavo po vrsticah lahko kontrolirate tako, da izvorno datoteko enkrat testno prevedete z opcijo \texttt{draft}, kar vam pokaže  predolge vrstice.



%\cleardoublepage
%\addcontentsline{toc}{chapter}{Literatura}

% če imaš težave poravnati desni rob bibliografije, potem odkomentiraj spodnjo vrstico
\raggedright

\printbibliography[heading=bibintoc,type=article,title={Članki v revijah}]

\printbibliography[heading=bibintoc,type=inproceedings,title={Članki v zbornikih}]

\printbibliography[heading=bibintoc,type=incollection,title={Poglavja v knjigah}]

% v zadnji verziji diplomskega dela običajno združiš vse tri vrste referenc v en sam seznam in
% izpustiš delne sezname
\printbibliography[heading=bibintoc,title={Literatura}]

\end{document}

